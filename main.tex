\input{~/macro.tex}
\usepackage{nicematrix,nccmath}
\title{物性物理学}
\author{20B01392 松本侑真}
\date{\today}
\begin{document}
\maketitle
\begin{abstract}

\end{abstract}
\tableofcontents
\newpage
\section{電子の振る舞いと加速定理}
まず、自由電子の場合、電子が従う方程式は
\begin{equation}
	\hat{H}\psi = -\frac{\hbar^2}{2m}\qty(\pdv[2]{x} + \pdv[2]{y} + \pdv[2]{z})\psi = E\psi
\end{equation}
である。この解として、
\begin{align}
	\psi_{\bm{k}}(\bm{r}) & = \frac{1}{\sqrt{V}}e^{i\bm{k}\vdot\bm{r}} = \frac{1}{\sqrt{V}}e^{i(k_xx + k_yy + k_zz)}                                                         \\
	E(\bm{k})             & = \frac{\hbar^2k^2}{2m}\quad\qty(\hat{\bm{P}}\psi = -i\hbar\grad{\psi} = -i\hbar\grad\frac{1}{\sqrt{V}}e^{i\bm{k}\vdot\bm{r}} = \hbar\bm{k}\psi)
\end{align}
を得る。また、一般に電子を物質波と見たときのエネルギーは
\begin{equation}
	E(\bm{k}) = h\nu(\bm{k}) = \hbar\omega(\bm{k})
\end{equation}
である。したがって、群速度(実際に観測される波の速度)は
\begin{equation}
	v_{g} = \pdv{\omega}{k} = \frac{1}{\hbar}\pdv{E}{k}\implies \bm{v}_g = \frac{1}{\hbar}\grad_{\bm{k}}E
\end{equation}
と表すことができる。
\subsection{加速定理}
加速定理とは、外力$\bm{F}$が波束に加わった際の変化を表す定理である。系には仕事が加わるため、エネルギー変化は
\begin{equation}
	\varDelta{}E = \bm{F}\vdot\bm{v}_g\varDelta{}t = \frac{\bm{F}}{\hbar}\grad_{\bm{k}}E(\bm{k})\varDelta{}t
\end{equation}
となる。一方で、
\begin{equation}
	\varDelta{}E = \pdv{E}{k_x}\varDelta{}k_x + \pdv{E}{k_y}\varDelta{}k_y + \pdv{E}{k_z}\varDelta{}k_z = \grad_{\bm{k}}{E}\vdot\varDelta\bm{k}
\end{equation}
より、
\begin{equation}
	\grad_{\bm{k}}{E}\vdot\varDelta\bm{k} = \frac{\bm{F}}{\hbar}\grad_{\bm{k}}E(\bm{k})\varDelta{}t
\end{equation}
となる。すなわち、
\begin{equation}
	\dv{k}{t} = \frac{\bm{F}}{\hbar}
\end{equation}
となる。これが加速定理である。この結果は、運動量が$\bm{P} = \hbar\bm{k}$と表されるため、
\begin{equation}
	\varDelta\bm{P} = \int\hbar\dd{\bm{k}} = \int \bm{F}\dd{t}
\end{equation}
となることを再現している。
\subsubsection*{自由電子の場合}
自由電子の場合に加速定理を適用して、$\bm{P} = m\bm{v}$を再現することを見る。
\begin{equation}
	\bm{v} = \bm{v}_g = \frac{1}{\hbar}\grad_{\bm{k}}{E} = \frac{\hbar\bm{k}}{m} \iff m\bm{v} = \hbar\bm{k} = \bm{P}
\end{equation}
となる。
\subsubsection*{一般の場合}
より一般には加速定理から運動方程式が成立することがわかる。まずは簡単のために一次元で考えると、
\begin{equation}
	\dv{v_g}{t} = \frac{1}{\hbar}\dv{t}\qty(\dv{E}{k}) = \frac{1}{\hbar}\qty(\dv[2]{E}{k})\dv{k}{t} = \frac{1}{\hbar^2}\qty(\dv[2]{E}{k})F
\end{equation}
となる。したがって、有効質量$m^*$を用いると、運動方程式
\begin{equation}
	m^*\dv{v_g}{t} = F,\quad{}m^* = \frac{1}{\hbar^2}\qty(\dv[2]{E}{k})^{-1}
\end{equation}
を得る。3次元の場合は以下のように拡張される:
\begin{equation}
	\dv{v_i}{t} = \frac{1}{\hbar}\dv{t}\qty(\dv{E}{k_i}) = \frac{1}{\hbar}\sum_{j}\qty(\pdv{}{k_i}{k_j}E)\dv{k_j}{t} = \frac{1}{\hbar^2}\sum_{j}\qty(\pdv{E}{k_i}{k_j})F_j\;。
\end{equation}
したがって、有効質量は$3\times3$行列(テンソル)となり、
\begin{equation}
	\qty(\frac{1}{m^*})_{ij} = \frac{1}{\hbar^2}\qty(\pdv{E}{k_j}{k_i})
\end{equation}
と表される。したがって運動方程式は
\begin{equation}
	\dv{\bm{v}}{t} =
	\begin{pNiceMatrix}
		\medmath{\frac{1}{\hbar^2}\qty(\pdv{E}{k_x}{k_x})} & \medmath{\frac{1}{\hbar^2}\qty(\pdv{E}{k_y}{k_x})} & \medmath{\frac{1}{\hbar^2}\qty(\pdv{E}{k_z}{k_x})} \\
		\medmath{\frac{1}{\hbar^2}\qty(\pdv{E}{k_x}{k_y})} & \medmath{\frac{1}{\hbar^2}\qty(\pdv{E}{k_y}{k_y})} & \medmath{\frac{1}{\hbar^2}\qty(\pdv{E}{k_z}{k_y})} \\
		\medmath{\frac{1}{\hbar^2}\qty(\pdv{E}{k_x}{k_z})} & \medmath{\frac{1}{\hbar^2}\qty(\pdv{E}{k_y}{k_z})} & \medmath{\frac{1}{\hbar^2}\qty(\pdv{E}{k_z}{k_z})}
	\end{pNiceMatrix}
	\bm{F}
\end{equation}
となる。

\section{Drudeモデル:微視的な電流を記述する古典的モデル}
Drudeモデルとは、有効質量が$m^*$、電荷が$-e$である場合の固体中の電子の電場$\bm{E}$の下における運動を記述するモデルである:
\begin{equation}
	m^*\dv{\bm{v}_e}{t} = -e\bm{E} -m^*\frac{\bm{v}_{e}}{\tau}\;。
\end{equation}
ここで、右辺第二項は不純物や格子振動による抵抗力を表す。抵抗力は電子の速度$\bm{v}_e$と有効質量に比例する。
また、$\tau$を時間の次元を持つ比例定数(緩和時間)として導入した。定常状態$\dv*{\bm{v}_d}{t}=0$では、方程式は次のように解くことができる:
\begin{equation}
	\bm{v}_d = -\frac{e\tau}{m^*}\bm{E} = -\mu\bm{E}\;。
\end{equation}
このときの速度$\bm{v}_{d}$をドリフト速度と呼ぶ。$\mu$を移動度(mobility)と呼び、固体中の電子の動きやすさを表す物理量である。

Drudeモデルで導かれる電流は、伝導度$\sigma$、抵抗率$\rho$とすると
\begin{equation}
	\bm{j} = -ne\bm{v}_d = ne\frac{e\tau}{m^*}\bm{E} = \frac{ne^2\tau}{m^*}\bm{E} = \sigma\bm{E} = \frac{1}{\rho}\bm{E}
\end{equation}
と表される。

\section{Boltzmann方程式:電子の速度分布を考慮して計算する}
古典的モデルであるDruleモデルでは、全ての電子が同じ速度$\bm{v}_d$で固体中を移動していると考えたが、
量子論では電子の速度は波数$\bm{k}$に依存し、速度分布が生じる。その分布を考慮するのがBoltzmann方程式である。

まず、時刻$t$における電子の個数分布関数を$f(\bm{r},\,\bm{k},\,t)$と置く。(電子の場合はフェルミオンなので、引数をエネルギーにすると$f$はFermi分布関数となる。)
固体中の電子の速度分布は格子振動や不純物などで変化する。このような現象が電子の散乱(scattering)である。電子が散乱されることによる分布関数の変化は、
\begin{equation}
	\dd{f} = \pdv{f}{t}\dd{t} + \sum_{\alpha=x,y,z}\pdv{f}{\alpha}\dd{\alpha} + \sum_{i=x,y,z}\pdv{f}{k_i}\dd{k}
	= \pdv{f}{t}\dd{t} + \qty(\grad{f})\vdot\dd{\bm{r}} + (\grad_{\bm{k}}f)\vdot\dd{\bm{k}} = \qty(\pdv{f}{t})_{\text{scat}}\dd{t}
\end{equation}
と表される。なお、最右辺の式は、電子の散乱が微小時間$\dd{t}$の間に引き起こされるときの分布関数の変化を形式的に表したものである。
以上より得られる以下の式をBoltzmann方程式と呼ぶ:
\begin{equation}
	\pdv{f}{t} + \qty(\grad{f})\vdot\dv{\bm{r}}{t} + (\grad_{\bm{k}}f)\vdot\dv{\bm{k}}{t} = \qty(\pdv{f}{t})_{\text{scat}}\;。
\end{equation}
また、電子の散乱は、電場が存在しない場合の平衡状態$f_0$へと分布関数$f$を戻そうとする働きがあると考える。その緩和時間を$\tau$とすると
\begin{equation}
	\qty(\pdv{f}{t})_{\text{scat}} = \frac{f_0-f}{\tau}
\end{equation}
と表される。

\subsection{電気伝導率について}
自由電子の電気伝導率をBoltzmann方程式を用いて導出する。
自由電子モデルでは、電子は空間一様分布をしているため、
分布関数は$\bm{r}$に依存しない:
\begin{equation}
	\grad{f} = 0\;。
\end{equation}
また、加える電場$\bm{E}$は時間に対して一定であるため、時間に対して定常状態となる:
\begin{equation}
	\pdv{f}{t} = 0\;。
\end{equation}
また、$\dv*{\bm{k}}{t} = \bm{F}/\hbar = -e\bm{E}/\hbar$であるため、Boltzmann方程式は
\begin{align}
	f(\bm{k}) = f_0(\bm{k}) + \frac{e\tau}{\hbar}\bm{E}\vdot\qty(\grad_k{f(\bm{k})})
	 & = f_0(\bm{k}) + \frac{e\tau}{\hbar}\bm{E}\vdot\qty(\grad_{\bm{k}}E)\dv{f}{\mathcal{E}} \notag \\
	 & = f_0(\bm{k}) + \frac{e\tau\hbar^2}{\hbar{}m^*}\bm{k}\vdot{\bm{E}}\dv{f}{\mathcal{E}}
\end{align}
となる。なお、$\bm{E}$が十分に小さく、Taylor近似が成立していると仮定すると、
\begin{equation}
	f(\bm{k}) =  f_0(\bm{k}) + \frac{e\tau}{\hbar}\bm{E}\vdot\qty(\grad_k{f_0(\bm{k})}) + \order{\bm{E}^2} \sim f_0\qty(\bm{k} + \frac{e\tau}{\hbar}\bm{E})
\end{equation}
と近似することもできる。こうして求まった分布関数を用いると、電流密度$\bm{j}$はスピン自由度と、波数空間で占める体積$1/(2\pi^3)$を考慮して
\begin{equation}
	\bm{j} = \frac{2}{(2\pi)^3}\int \dd{\bm{k}} (-e\bm{v}(\bm{k}))f(\bm{k})
\end{equation}
と求まる。\footnote{周期境界条件では$kL = 2n\pi$であるため、波数空間内で1つの$\bm{k}$が占める体積は$V/(2\pi)^3$である。}
自由電子の速度ベクトルは${\hbar\bm{k}}/{m^*}$と表されるため、$x$方向のみに電場が印加しているとすると、
\begin{equation}
	j_x = -\frac{2}{(2\pi)^3}\int \dd{\bm{k}} \qty(e\frac{\hbar{}k_x}{m^*})\qty(f_0(\bm{k}) + \frac{e\tau\hbar{}k_xE_x}{m^*}\dv{f}{\mathcal{E}})
\end{equation}
となる。
しかし、$f_0$は平衡状態であり、$\bm{k}$に関する偶関数となっていることに注意すると、$\bm{j}=0$となってしまう。
したがって、
\begin{equation}
	j_x = -\frac{2}{(2\pi)^3}\int \dd{\bm{k}} \qty(e\frac{\hbar{}k_x}{m^*})\frac{e\tau\hbar{}k_xE_x}{m^*}\dv{f}{\mathcal{E}}
	= -\frac{2e^2\tau\hbar^2}{(2\pi)^3{m^*}^2}E_x\int4\pi{}k^2\dd{k}k_x^2\dv{f}{\mathcal{E}}
\end{equation}
と変形できる。$k_x=k_y=k_z$を仮定すると、$k_x^2 = k^2/3$となり、
\begin{equation}
	\mathcal{E} = \frac{\hbar^2k^2}{2m^*} \implies 2k\dd{k} = \frac{2m^*}{\hbar^2}\dd{\mathcal{E}} \implies \dd{k} = \frac{2m^*}{\hbar^2}\qty(\frac{2m^*\mathcal{E}}{\hbar^2})^{-1/2}\dd{\mathcal{E}}
\end{equation}
を用いると、
\begin{equation}
	j_x = -\frac{2e^2\tau\hbar^2}{(2\pi)^3{m^*}^2}E_x\int\frac{4}{3}\pi{}\qty(\frac{2m^*\mathcal{E}}{\hbar^2})^{3/2}\frac{2m^*}{\hbar^2}\dd{\mathcal{E}}\dv{f}{\mathcal{E}}
	= -\frac{2e^2\tau\hbar^2}{(2\pi)^3{m^*}^2}E_x\int\frac{4}{3}\pi{}\qty(\frac{2m^*}{\hbar^2})^{5/2}\mathcal{E}^{3/2}\dd{\mathcal{E}}\dv{f}{\mathcal{E}}
\end{equation}
となる。さらに、$f$はフェルミ分布関数であるため、$\dv*{f}{\mathcal{E}} \sim -\delta(\mathcal{E}-\mathcal{E}_{F})$が成立し、
\begin{equation}
	j_x = \frac{2e^2\tau\hbar^2}{(2\pi)^3{m^*}^2}E_x\int\frac{2}{3}\pi{}\qty(\frac{2m^*}{\hbar^2})^{5/2}\mathcal{E}^{3/2}\dd{E}\delta(\mathcal{E}-\mathcal{E}_F)
	= \frac{e^2\tau}{3\pi^2{m^*}}\qty(\frac{2m^*}{\hbar^2})^{3/2}\mathcal{E}_F^{3/2}E_x
\end{equation}
となる。フェルミエネルギー$\mathcal{E}_F$は、全粒子数$n$を用いて
\begin{equation}
	\mathcal{E}_F = \frac{\hbar^2}{2m^*}(3\pi^2n)^{2/3} = \frac{\hbar^2k_{F}^2}{2m^*}
\end{equation}
と表されるため、
\begin{equation}
	j_x = \frac{e^2\tau{}n}{m^*}E_x \implies \sigma = \frac{e^2\tau{}n}{m^*}
\end{equation}
を得る。この結果は古典的なDrudeモデルで得られる結果と一致する。




\section{温度勾配による電流と熱流}

\end{document}