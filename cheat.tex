\input{~/macro.tex}

\begin{document}
加速定理$\dv{k}{t} = F/\hbar$は、エネルギー変化を$\bm{F}\vdot\bm{v}_{g}\varDelta{}t = \grad_{\bm{k}}E\vdot\varDelta\bm{k}$であらわす。
有効質量方程式は以下
\begin{equation}
	\dv{v_g}{t} = \frac{1}{\hbar}\dv{t}\dv{\mathcal{E}}{k} = \frac{1}{\hbar}\dv{k}{t}\dv[2]{\mathcal{E}}{k} = \frac{F}{\hbar^2}\dv[2]{\mathcal{E}}{k}
\end{equation}
Drudeモデル
\begin{equation}
	m^*\dv{\bm{v}}{t} = q(\bm{E}+\bm{v}\cross\bm{B}) - \frac{m\bm{v}}{\tau}
\end{equation}
Boltzmann方程式を用いると、電流の模型は以下のようになる:
\begin{equation}
	f(\bm{k}) =  f_0(\bm{k}) + \frac{e\tau}{\hbar}\bm{E}\vdot\qty(\grad_k{f_0(\bm{k})}) + \order{\bm{E}^2} \sim f_0\qty(\bm{k} + \frac{e\tau}{\hbar}\bm{E})
\end{equation}
\begin{equation}
	\bm{j} = \frac{2}{(2\pi)^3}\int \dd{\bm{k}} (-e\bm{v}(\bm{k}))f(\bm{k})
\end{equation}
自由電子モデルでは、電子は空間一様分布をしているため、分布関数は$\bm{r}$に依存しないこと$\grad{f} = 0$に注意。
自由電子の速度ベクトルは${\hbar\bm{k}}/{m^*}$と表されるため、$x$方向のみに電場が印加しているとすると、
\begin{equation}
	j_x = -\frac{2}{(2\pi)^3}\int \dd{\bm{k}} \qty(e\frac{\hbar{}k_x}{m^*})\qty(f_0(\bm{k}) + \frac{e\tau\hbar{}k_xE_x}{m^*}\dv{f}{\mathcal{E}})
\end{equation}
となる。
しかし、$f_0$は平衡状態であり、$\bm{k}$に関する偶関数となっていることに注意すると、$\bm{j}=0$となってしまう。
したがって、
\begin{equation}
	j_x = -\frac{2}{(2\pi)^3}\int \dd{\bm{k}} \qty(e\frac{\hbar{}k_x}{m^*})\frac{e\tau\hbar{}k_xE_x}{m^*}\dv{f}{\mathcal{E}}
	= -\frac{2e^2\tau\hbar^2}{(2\pi)^3{m^*}^2}E_x\int4\pi{}k^2\dd{k}k_x^2\dv{f}{\mathcal{E}}
\end{equation}
と変形できる。$k_x=k_y=k_z$を仮定すると、$k_x^2 = k^2/3$となり、
\begin{equation}
	\mathcal{E} = \frac{\hbar^2k^2}{2m^*} \implies 2k\dd{k} = \frac{2m^*}{\hbar^2}\dd{\mathcal{E}} \implies \dd{k} = \frac{2m^*}{\hbar^2}\qty(\frac{2m^*\mathcal{E}}{\hbar^2})^{-1/2}\dd{\mathcal{E}}
\end{equation}
を用いると、
\begin{equation}
	j_x = -\frac{2e^2\tau\hbar^2}{(2\pi)^3{m^*}^2}E_x\int\frac{4}{3}\pi{}\qty(\frac{2m^*\mathcal{E}}{\hbar^2})^{3/2}\frac{2m^*}{\hbar^2}\dd{\mathcal{E}}\dv{f}{\mathcal{E}}
	= -\frac{2e^2\tau\hbar^2}{(2\pi)^3{m^*}^2}E_x\int\frac{4}{3}\pi{}\qty(\frac{2m^*}{\hbar^2})^{5/2}\mathcal{E}^{3/2}\dd{\mathcal{E}}\dv{f}{\mathcal{E}}
\end{equation}
となる。さらに、$f$はフェルミ分布関数であるため、$\dv*{f}{\mathcal{E}} \sim -\delta(\mathcal{E}-\mathcal{E}_{F})$が成立し、
\begin{equation}
	j_x = \frac{2e^2\tau\hbar^2}{(2\pi)^3{m^*}^2}E_x\int\frac{2}{3}\pi{}\qty(\frac{2m^*}{\hbar^2})^{5/2}\mathcal{E}^{3/2}\dd{E}\delta(\mathcal{E}-\mathcal{E}_F)
	= \frac{e^2\tau}{3\pi^2{m^*}}\qty(\frac{2m^*}{\hbar^2})^{3/2}\mathcal{E}_F^{3/2}E_x
\end{equation}
となる。フェルミエネルギー$\mathcal{E}_F$は、全粒子数$n$を用いて
\begin{equation}
	\mathcal{E}_F = \frac{\hbar^2}{2m^*}(3\pi^2n)^{2/3} = \frac{\hbar^2k_{F}^2}{2m^*}
\end{equation}
と表されるため、
\begin{equation}
	j_x = \frac{e^2\tau{}n}{m^*}E_x \implies \sigma = \frac{e^2\tau{}n}{m^*}
\end{equation}
Hall効果は
\begin{equation}
	\begin{dcases}
		0 & = q\qty(E_x + v_yB_z) - \frac{m^*v_x}{\tau} \\
		0 & = q\qty(E_y - v_xB_z) - \frac{m^*v_y}{\tau}
	\end{dcases} \implies
	\begin{pmatrix}
		\frac{m^*}{\tau} & {-qB_z}            \\
		{qB_z}           & {\frac{m^*}{\tau}}
	\end{pmatrix}
	\begin{pmatrix}
		v_x \\
		v_y
	\end{pmatrix}
	= \begin{pmatrix}
		qE_x \\ qE_y
	\end{pmatrix}
\end{equation}
\begin{align}
	v_x & = \frac{q\tau/m^*}{1+(qB_z\tau/m^*)^2}\qty(E_x + \frac{qB_z\tau}{m^*}E_y) \\
	v_y & = \frac{q\tau/m^*}{1+(qB_z\tau/m^*)^2}\qty(E_y - \frac{qB_z\tau}{m^*}E_x)
\end{align}
を得る。したがって、サイクロトロン各周波数$\omega_c = qB_z/m^*$を用いると、
\begin{align}
	j_x & = nqv_x = \frac{nq^2\tau/m^*}{1+(qB_z\tau/m^*)^2}\qty(E_x + \frac{qB_z\tau}{m^*}E_y) = \frac{\sigma}{1+\omega_c^2\tau^2}\qty(E_x + \omega_c\tau{}E_y) \\
	j_y & = nqv_y = \frac{nq^2\tau/m^*}{1+(qB_z\tau/m^*)^2}\qty(E_y + \frac{qB_z\tau}{m^*}E_x) = \frac{\sigma}{1+\omega_c^2\tau^2}\qty(E_y - \omega_c\tau{}E_x)
\end{align}
温度勾配による熱流:
$f(\bm{k},\,T)$において$\grad_{r}T\neq 0$の状況を考える。Boltzmann方程式は
\begin{equation}
	\pdv{f}{t} + \grad_{\bm{r}}f\vdot\dv{\bm{r}}{t} + \grad_{\bm{k}}\dv{\bm{k}}{t} = -\frac{f-f_0}{\tau}
\end{equation}
であるが、定常状態を考えると、
\begin{equation}
	f(\bm{k}) = f_0(\bm{k}) - \tau\grad_{\bm{k}}f\vdot\dv{\bm{k}}{t} - \tau\grad_{\bm{r}}f\vdot\dv{\bm{r}}{t}
\end{equation}
となる。ここで、
\begin{equation}
	\dv{\bm{k}}{t} = -\frac{e\bm{E}}{\hbar},\,\grad_{\bm{r}}f = \pdv{f}{T}\grad_{\bm{r}}T
\end{equation}
を代入すると、
\begin{equation}
	f(\bm{k}) = f_0(\bm{k}) + \frac{e}{\hbar}\tau\bm{E}\vdot\grad_{\bm{k}}f(\bm{k}) - \tau\pdv{f(\bm{k})}{T}\bm{v}\vdot\grad_{\bm{r}}T
\end{equation}
となる。電流の表式
\begin{equation}
	\bm{j} = -\frac{2}{(2\pi)^3}\int\dd{\bm{k}}e\bm{v}(\bm{k})f(\bm{k})
\end{equation}
に代入すると、
\begin{equation}
	\bm{j} = -\frac{2e}{(2\pi)^3}\int\dd{\bm{k}}\bm{v}(\bm{k})\qty[f_0(\bm{k}) + \frac{e}{\hbar}\tau\bm{E}\vdot\grad_{\bm{k}}f(\bm{k}) - \tau\pdv{f(\bm{k})}{T}\bm{v}\vdot\grad_{\bm{r}}T]
\end{equation}
となる。$f_0(\bm{k})$は電場が印加していないときの分布関数であるため、全積分は$0$となる。
また、$\pdv*{T}{x}$の一次までの寄与を考える際には、右辺第三項で$f\to{}f_0$とすればよい。

\begin{equation}
	\grad_{\bm{r}}T = \mqty(\dv{T}{x} & 0 & 0),\,\bm{E}=\bm{0}
\end{equation}
の場合を考えると、
\begin{equation}
	j_x = e\int_0^{\infty}\dd{\mathcal{E}}D(\mathcal{E})\tau(\mathcal{E})\frac{v^2}{3}\pdv{f_0}{x}\pdv{T}{x}
	\approx \frac{1}{3}e\tau_{\text{F}}v^2_{\text{F}}\int_0^{\infty}\dd{\mathcal{E}}\qty(D(\mathcal{E}_{\text{F}}) + D'(\mathcal{E_{\text{F}}}))
	\times\frac{\exp(\frac{\mathcal{E}-\mathcal{E}_F}{k_{\text{B}}T})}{\qty[\exp(\frac{\mathcal{E}-\mathcal{E}_F}{k_{\text{B}}T}) + 1]^2}\frac{\mathcal{E}-\mathcal{E}_{\text{F}}}{k_{\text{B}}^2T}\pdv{T}{x}
\end{equation}
となる。なお、
\begin{equation}
	f_0 = \frac{1}{\exp(\frac{\mathcal{E}-\mathcal{E}_{\text{F}}}{k_{\text{B}}T})+1}
\end{equation}
を用いた。さらに、$D(\mathcal{E}_{\text{F}})=0$であるため、
\begin{equation}
	\int_0^\infty\dd{\mathcal{E}}D(\mathcal{E}_{\text{F}})=0,\quad x = \frac{\mathcal{E}-\mathcal{E}_{\text{F}}}{k_{\text{B}}T}
\end{equation}
とすると、
\begin{equation}
	j_x = \frac{1}{3}e\tau_{\text{F}}v^2_{\text{F}}D'(\mathcal{E}_{\text{F}})\pdv{T}{x}\frac{(k_{\text{B}}T)^2}{T}\int_{-\infty}^{\infty}\frac{e^xx^2}{(e^x+1)}\dd{x}
	= \frac{\pi^2}{9}e\tau_{\text{F}}v^2_{\text{F}}D'(\mathcal{E}_{\text{F}})k^2_{\text{B}}T\pdv{T}{x} \coloneqq -\mathcal{L}^{12}\pdv{T}{x}
\end{equation}
となる。熱力学$Q=T\dd{S} = \dd{U}-\mu\dd{n} = \dd{U}-\mathcal{E}_{\text{F}}\dd{n}$との比較から、熱流密度は
\begin{equation}
	\bm{j}_{Q} = \bm{j}_{\mathcal{E}} - \mathcal{E}_{\text{F}}\bm{j}_n
\end{equation}
のようになると考えられる。より一般的には、交差的な項$\mathcal{L}^{12},\,\mathcal{L}^{21}$が混合して流れが求められる:
\begin{align}
	\bm{j}   & = \mathcal{L}^{11}\bm{E} + \mathcal{L}^{12}\qty(-\grad_{\text{r}}{T})\;, \\
	\bm{j}_Q & = \mathcal{L}^{21}\bm{E} + \mathcal{L}^{22}\qty(-\grad_{\text{r}}{T})\;。
\end{align}
\begin{align*}
	\bm{j}_Q & = \bm{j}_{\mathcal{E}}-\mathcal{E}_{\text{F}}\bm{j}_n
	= \frac{2}{(2\pi)^3}\int\dd{\bm{k}}\mathcal{E}(\bm{k})\bm{v}(\bm{k})f(\bm{k}) - \mathcal{E}_{\text{F}}\frac{2}{(2\pi)^3}\int\dd{k}\bm{v}(\bm{k})f(\bm{k})                      \\
	         & = \frac{2}{(2\pi)^3}\int\dd{\bm{k}}(\mathcal{E}-\mathcal{E}_{\text{F}})v(k)\qty[f_0 + \frac{e\tau}{\hbar}\bm{E}\vdot\grad_kf_0 - \tau\pdv{f}{T}\bm{v}\vdot\grad_rT]
	= \frac{2}{(2\pi)^3}\int\dd{\bm{k}}(\mathcal{E}-\mathcal{E}_{\text{F}})v^2_x\tau\bm{v}\dv{f_0}{T}\qty(-\pdv{T}{x})
\end{align*}
\begin{align*}
	j_Q & = \int\dd{\mathcal{E}}D(\mathcal{E})(\mathcal{E}-\mathcal{E}_{\text{F}})\frac{v^2\tau}{3}\frac{\exp(\frac{E-E_F}{kT})}{\qty(\exp(\frac{E-E_F}{kT})+1)^2}\frac{E-E_F}{kT^2}\qty(-\pdv{T}{x})
	\\
	    & =\frac{1}{3}v_F^2D(E_F)k^2T\qty(-\pdv{T}{x})\int_{-\infty}^{\infty}\frac{x^2e^2}{(e^x+1)^2}\dd{x} = \frac{1}{3}v_F^2\tau_F\qty(\frac{\pi^2}{3}D(E_F)k^2T)\qty(-\pdv{T}{x})
	= -\frac{1}{3}v_F^2\tau_FC_V\pdv{T}{x}
\end{align*}
ここで、$E_F = \frac{1}{2}m^*v_F^2$と、
\begin{equation}
	\dv{E_F}{n} = \dv{n}\frac{\hbar^2}{2m^*}(3\pi^2n)^{2/3} = \dv{n}Cn^{2/3} = \frac{2}{3}\frac{Cn^{2/3}}{n} = \frac{2E}{3n} \implies D(E_F) = \frac{dn}{dE} = \frac{3}{2}\frac{n}{E_F}
\end{equation}
より、
\begin{equation}
	j_Q = \frac{1}{3}v_F^2\tau_F\qty(\frac{\pi^2}{3}D(E_F)k^2T)\qty(-\pdv{T}{x}) = -\lambda_E\pdv{T}{x} ,\,\lambda_E = \frac{1}{3}v_F^2\tau_F\frac{\pi^2}{2}nk\frac{kT}{m^*v_F^2/2} = \frac{n\tau\pi^2k^2T}{3m^*}
\end{equation}
電流が流れていないとき、$E = \mathcal{L}^{12}/\mathcal{L}^{11}\pdv{T}{x}$となり、係数をSeebeck係数と呼ぶ。ループに沿って温度一定のとき、$j_Q = \mathcal{L}^{12}/\mathcal{L}^{11}j = \Pi{}j$となり、Peltier係数と呼ぶ。

有効状態密度:$\mathcal{E}-\mu\sim\SI{500}{meV} \gg kT\sim\SI{30}{meV}$のとき、フェルミ分布関数が近似できる:
\begin{equation}
	f = \frac{1}{e^{{(E-E_F)}/{kT}} +1} = \exp(-\frac{\mathcal{E}-\mu}{kT})
\end{equation}
ここで、$E = \hbar^2k^2/2m^*$より、
\begin{equation}
	D(E)\dd{E} = 2\times\frac{L^3}{(2\pi)^3}4\pi{}k^2\dd{k} = \frac{V}{2\pi^2}\qty(\frac{2m^*}{\hbar^2})^{3/2}E^{1/2}\dd{E}
\end{equation}
ゆえ、伝導帯の電子数$n_e$は、$E-\mu = E-E_c + E_c -\mu$に注意して
\begin{equation}
	n_e = \frac{1}{2\pi^2}\qty(\frac{2m^*_c}{\hbar^2})^{3/2}\int_{E_c}^{\infty}(E-E_c)^{1/2}\exp{-\frac{E-\mu}{kT}}\dd{E}
\end{equation}
$x = E-E_c/kT$として
\begin{equation}
	n_e = \frac{1}{2\pi^2}\qty(\frac{2m^*_c}{\hbar^2})^{3/2}(kT)^{3/2}\exp{\frac{E_c-\mu}{kT}}\Gamma(3/2) = 2\qty(\frac{m^*_ckT}{2\pi\hbar^2})^{3/2}\exp(-\frac{E_c-\mu}{kT})
\end{equation}
となる。真正半導体のときは$n_e=n_h$なので、かけてルート取ると$\mu$消えて簡単に計算することができる。一方で、直接比較すると$\mu$が残って
\begin{equation}
	\mu ~ \frac{E_c+E_V}{2} + \frac{3}{4}kT\log\qty(\frac{m^*_V}{m^*_c})
\end{equation}




\end{document}